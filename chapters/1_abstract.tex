\documentclass[../main]{subfiles}
\begin{document}
\chapter*{Abstract}
Software testing is a crucial part of the software engineering process. 
It is widely used in the industry.
A part of software testing is building test suites which contain unit tests.
These unit tests are written by developers. 
As projects grow the test suite grows along.
Maintaining and monitoring these test suites is important because they influence the cost of maintenance.
Monitoring the quality of test code has shown that tests with lower quality leads to more defect-prone production code.
\newline
Measuring the effectiveness of test suites detecting faults is part of the monitoring process.
A technique to measure the test effectiveness is mutation testing.
Mutation testing is a computationally expensive technique that requires the ability to run tests.
Different techniques have been proposed to reduce the cost of mutation testing.
One of these techniques is the clustering of mutants.
By clustering mutants we can execute less mutants to reduce the cost.
\newline
We conducted two experiment in which we cluster the mutants with two different techniques.
Before conducting the experiment we conducted a state of the art comparison between three mutation testing tools for Java.
Based on this research we selected a mutation testing tool to use in our experiments.
\newline
We started our experiment by identifying characteristics to represent the mutants.
For the first experiment we used hierarchical clustering and in the second experiment we trained the \acrlong{fcm} model.
We calculated a weighted mutation score and compared this with the mutation score of a full set of mutants executed.
Results show that with hierarchical clustering we can reduce the amount of mutants executed while maintaining the effectiveness.
The clusters generated by the machine learning model were less accurate and showed a significant decrease in effectiveness.
The machine learning model didn't perform as well as the hierarchical clustering.
However, more work is needed in finding the right parameters for the \acrlong{fcm} algorithm. 
\end{document}