\documentclass[../main]{subfiles}
\begin{document}

\chapter{Background}
\label{ch:background}
This chapter will present the necessary background information for this thesis. First, we define some basic terminology that will be used throughout this thesis.
Next, we will compare existing mutation testing tools. We start of by reviewing existing literature and then extend the comparison. \todo{extend this later on?}
\section{Terminology}
\todo{do this when we finished experiments and all}
\todo{subsuming mutants}
\todo{levensteihn distance}
\section{Project selection}
We chose two main requirements for selecting software projects; the projects should have a test suite and the test suite should not contain failing tests.
We started out by selecting projects that were also used in other research within the mutation testing and testing domain\cite{Pizzoleto2019,Yu2019PossibilityScope,Wei2021SpectralTesting, Zhang2019PredictiveTesting, Chen2018SpeedingStudy, Laurent2017AssessingPIT}.
This resulted in six projects that satisfied the requirements.
To extend our sample of projects we selected projects from the first six pages of the most popular Java projects on GitHub\footnote{\url{https://github.com/search?p=7&q=language\%3Ajava+stars\%3A\%3E10000&type=Repositories}}.
From this selection we filtered out all the projects that contained failing tests that weren't libraries or applications.
We ended up with a total sample of 21 projects. 
These projects consists of Java libraries, Android libraries and Java applications.

\section{Clustering algorithms}

\end{document}