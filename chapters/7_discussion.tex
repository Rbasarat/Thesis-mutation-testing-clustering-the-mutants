\documentclass[../main]{subfiles}
\begin{document}


\chapter{Discussion}
\label{ch:discussion}
In this chapter, we discuss the results of our experiments on clustering the mutants.
We will first discuss the results of the first experiment. 
For the first experiment will discuss the all the results together and compare the results between the variants.
Secondly we will discuss the results of the second experiment.
Thirdly we will discuss how the results both experiments compare to each other.
And last we will discuss the practicality of our research.s

\section{RQ 1: Identifying and clustering mutant characteristics}
The maximum deviation from the original mutation score, for all variants(with and without Levenshtein distance) of the experiment, are below the statistical 
As hypothesised we observe that it is possible to cluster mutants with the set of characteristics we identified while maintaining effectiveness and reducing the amount executed.
The hypothesis holds for all the projects and variants of the experiment.
Most of the projects had a deviation of less than 1\% for all variations.

\begin{finding}
    With a reduction up to 75\% of the total amount of mutants we can still maintain effectiveness.
\end{finding}
In both variants we observe a reduction in average accuracy when reducing the amount of clusters.
This reductions is 0.1\% for clustering with all characteristics and 0.04\% for clustering without Levenshtein distance.
We can also see that the average cluster accuracy is lower when reducing the amount executed.
\begin{finding}
    The more you reduce the amount of mutants executed the more accuracy is lost.
\end{finding}
The results show that the average accuracy for reductions of 25\% and 50\% is the same.
This observation shows true for both variants of the experiment which is remarkable.
\begin{finding}
    In our data set it is safe to reduce the mutants by half without reducing the accuracy.
\end{finding}
It is also remarkable to see the min and max accuracy of the all clusters are 50\% to 100\% respectively.
Based on these results we can state the following:
\begin{itemize}
    \item For every project there was at least one cluster that consisted only of killed or survived mutants.
    \item For every project there was at least one cluster that evenly divided between killed and survived mutants.
\end{itemize}
With the exception of all Google Auto Service and Commons Text with a reduction of 25\% and all characteristics all projects showed an average accuracy above 90\%.
The combination of hierarchical clustering and the identified characteristics results in high accuracy clusters.
As logic dictates we expect that a higher accuracy means a score closer to that of the original.
However we do not observe this behaviour in our results.
This is due to our random sampling method.
If we would select a mutant in the majority of each cluster the accuracy per cluster would correlate to the deviation in mutation score.
Unfortunately we do not know the majority of a cluster as we do now know if a mutant survives or is killed before execution.
\begin{finding}
    Random sampling a mutant per cluster may result in a bigger deviation between original score and weighed score.
\end{finding}
As stated in Chapter \ref{ch:results_rq1} the calculation of the Levenshtein distance took a long time.
For this reason we repeated the experiment without the Levenshtein distance.
Comparing the results between these variants we can observe a max deviation of 0.05\% between average difference of the mutation score of.
In 66\% of the results displayed in Chapter \ref{ch:results_rq1} clustering without the Levenshtein distance is more accurate.
The biggest difference measured between all the projects and their reductions is 2.13\% where of clustering without Levenshtein distance was more accurate.
\begin{finding}
    While clustering without the Levenshtein distance was more accurate in our results, the Levenshtein distance characteristic does not significantly increase or decrease the cluster accuracy.
\end{finding}




The application of our research is limited to all software projects written in the language Java and the limitations of our mutation testing tool PIT\cite{pit}.
This scope is significantly larger than most of the research done in the domain of mutation clustering(Chapter \ref{ch:related_work}).

\section{Threats to validity}
What could affect the validity of your research? Think of pitfalls of your research method, experimental setup, interpreting the results.
- pit generated mutants -> it's hard to know if we can cluster ALL possible mutants
- Selected sources
- equivalent mutants


\end{document}