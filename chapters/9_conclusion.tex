\documentclass[../main]{subfiles}
\begin{document}

\chapter{Conclusion}
\label{ch:conclusion}
Our research started with comparing three mutation testing tools for Java and found that the state of the art was outdated.
We replicated the state of the art research with current data available on the mutation testing tools.
The state of art showed that two out of three mutation testing tools were not actively maintained anymore.
The mutation testing tool PIT scored best on the metrics measured in our replication study, this is also the tool we decided to use for our research.
\newline
We created a list of characteristics based on existing research. 
We then extended this list by identifying characteristic with logic and data available.
We used the identified characteristics to represent the mutants.
\newline
We clustered the mutants with two different clustering techniques; hierarchical clustering and by training a machine learning clustering model.
The hierarchical clustering showed that we can cluster mutants while maintaining effectiveness.
We achieved reductions of up to 75\% of the total amount of mutants without significantly reducing the accuracy.
\newline
The machine learning model showed a significant reduction in effectiveness.
We trained the machine learning model with parameters that were optimal according to existing research. 
We also used a machine learning model, specialized in calculating the number of clusters for fuzzy clustering, to calculate the number of clusters.
While the minimum clustering accuracy was higher than that of the hierarchical clustering, the maximum cluster accuracy was lower than that of the hierarchical clustering.
The cluster accuracy had a greater impact on the clusters of the machine learning model because there was a relatively small number of clusters.
Our research proves that it is possible to determine the means of domains as the centre of a cluster within a programming language without narrowing down the scope to specific pieces of code.
It can be applied to all Java projects that can be run with PIT. 
Which is a larger scope than in existing mutation clustering research.

\section{Future work}
\label{sec:future_work}
\subsubsection{Optimizing parameters of machine learning model}
We trained the machine learning model with specific parameters guided by science. 
These proved to be non optimal for our purpose.
This research can be extended by finding the optimal parameters for the \acrshort{som} clustering algorithm.
The results showed a higher minimum accuracy which can indicate that there is potential for clustering the mutants with the machine learning model.
Further research and fine-tuning is required to find out if the model can be trained to achieve more accurate scores.

\subsubsection{Training strategy}
Our approach for training the model proved to be non optimal.
During the training of the model we experienced that the more clusters you select the more memory is needed for training.
This research did not have sufficient hardware available.
A different approach in training strategy may be explored to mitigate the hardware limitation.

\subsubsection{Training a single machine learning model}
For this researched we trained a model for each source project. 
A different way is to train one machine learning model with the total set of mutants.
The model could be continuously trained each time a new source is added.
Further research is recommended in this direction as it will improve the applicability of this research.

\subsubsection{Different sets of characteristics}
We have proven that this set of characteristics can achieve accurate results in combination with hierarchical clustering.
It would be valuable to find out if it possible to reduce or expand the set of characteristics and still maintain accurate results.
Further experimentation can be done with different sets of characteristics.
\end{document}