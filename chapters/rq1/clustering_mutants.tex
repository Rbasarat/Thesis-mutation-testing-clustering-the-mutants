\documentclass[../../main]{subfiles}
\begin{document}
\section{Clustering mutants, a white box approach}
\label{ch:clustering_characteristics}
We identified characteristics to represent mutants in Chapter \ref{ch

\subsection{Levenshtein distance on Java byte code}
A mutant is a piece of code that differs in a predefined way from its parent.
Java source code compiles into Java byte code.
During the compilation of Java code certain context is abstracted away\cite{byteCodeEngineering}.
Optimisations are applied which also changes the Java byte code\cite{byteCodeEngineering}.
As a result the textual similarity between a Java code mutant and its parent is different from the similarity between a Java byte code mutant and its parent.
While the textual similarity may be different between the Java code and Java byte code the functionality remains the same.
Java byte code reflects more of the semantic nature of the source code than the source code itself does. 
PIT generates and executes mutants on byte code level\cite{pitestBytecode}. 
In other words PIT executes the unit tests of a source against the byte code of a mutant.
\newline
By calculating the distance on byte code we filter out the context that is present in Java code.
This gives us a Levenshtein distance that represents more of the semantic difference, between a mutant and parent, than calculating the distance for Java code.
For this characteristic we will use the Levenshtein distance between a Java byte code mutant and its parent.
\end{document}