\documentclass[../../main]{subfiles}
\begin{document}
% Because we use sections but these sections should be interpreted as chapters for rq1 and rq2 sectioning.
\clearpage
\section{Experiment design}
In this chapter we will design an experiment to answer our second research question.
We hypothesise that we train a machine learning model with the set of characteristics we identified
while maintaining effectiveness and reducing the amount executed when executing one mutant from each cluster that is randomly selected.
Our goal is to achieve a mutation score that is as close as possible to the mutation score of a full set of executed mutants.
\newline
\todo{finish this}

\subsection{Selecting a machine learning model}
Our starting point for selecting a machine learning model is the survey done by K.L. Du\cite{Du2010Clustering:Approach}.
We start with a process of elimination to filter the machine learning models named in chapter \ref{ch:background}.
\newline
supervised learning\todo{elaborate why no supervised}
\newline
The moutain clustering model is effective method for estimating the number of clusters\cite{Du2010Clustering:Approach}. 
The goal of this experiment is to cluster mutants, while finding out the number of clusters we need might be valuable it is not what we want to achieve. Therefore we will not select mountain clustering.
Subtractive clustering is a modified form of mountain c\cite{Du2010Clustering:Approach}. 
\newline

\subsection{Training method}

\subsection{Validation}

\end{document}