\documentclass[conference]{IEEEtran}
\IEEEoverridecommandlockouts
% The preceding line is only needed to identify funding in the first footnote. If that is unneeded, please comment it out.
% \usepackage{cite}
\usepackage{amsmath,amssymb,amsfonts}
\usepackage{algorithmic}
\usepackage{graphicx}
\usepackage{textcomp}
\usepackage{xcolor}
\usepackage{makecell}
\usepackage{caption}
\usepackage{subcaption}
\usepackage{hyperref}

\usepackage[flushleft]{threeparttable}
\usepackage{tabularx}

\usepackage[disable]{todonotes}
\usepackage{enumitem}

\usepackage[backend=biber, style=ieee, citestyle=numeric-comp,uniquelist=false, maxcitenames=2, maxbibnames=9, mincitenames=1]{biblatex}
\bibliography{../main}
\bibliography{../custom}

%% uncomment this and change compiler to xelatex %%% Thomas uncomment over here
%\usepackage{fontspec}
%\setmainfont{OpenDyslexic}

\usepackage{listings}
\lstset{
	basicstyle=\footnotesize,        % the size of the fonts that are used for the code
	breakatwhitespace=false,         % sets if automatic breaks should only happen at whitespace
	breaklines=true,                 % sets automatic line breaking
	captionpos=b,                    % sets the caption-position to bottom
	frame=single,                    % adds a frame around the code
	language=Java,                 % the language of the code
	keywordstyle=\bf,
	tabsize=2                       % sets default tabsize to 2 spaces
}


\def\BibTeX{{\rm B\kern-.05em{\sc i\kern-.025em b}\kern-.08em
    T\kern-.1667em\lower.7ex\hbox{E}\kern-.125emX}}

\def\code#1{\texttt{#1}}
\begin{document}

\title{Mutation testing: Clustering the mutants}


\author{
\IEEEauthorblockN{Ana Oprescu}
\IEEEauthorblockA{\textit{Parallel Computing Systems} \\
\textit{University of Amsterdam}\\
Amsterdam, the Netherlands \\
a.m.oprescu@uva.nl}

\and
\IEEEauthorblockN{Rasjaad Basarat}
\IEEEauthorblockA{\textit{Parallel Computing Systems} \\ %% TODO: CHANGE THIS
\textit{University of Amsterdam}\\
Amsterdam, the Netherlands \\
rasjaad@nivero.io}
\and
\IEEEauthorblockN{Thomas Biesaart}
\IEEEauthorblockA{\textit{Parallel Computing Systems} \\ %% TODO: CHANGE THIS
Amsterdam, the Netherlands \\
thomas.biesaart@protonmail.com}
}

\maketitle
% \todo[inline]{FROM REVIEWER: Make sure to be consistent with singular/plural (e.g. mutation operators will result in mutants, not just a mutant)}
% \todo[inline]{FROM REVIEWER:  I don't quite follow the discussion of the execution time problem. After all, even in a decentralized approach I can run multiple approaches on the same hardware and thus make a fair comparison. I suppose the point you're trying to make is comparison *across multiple papers/studies"? Please clarify.}
\begin{abstract}
Mutation testing is a computationally expensive technique that requires the ability to run tests.
Different techniques have been proposed to reduce the cost of mutation testing.
One of these techniques is the clustering of mutants.
By clustering mutants we can execute less mutants to reduce the cost.
Different techniques have been proposed to reduce the cost of mutation testing.
One of these techniques is the clustering of mutants.
Our research consists of two parts in which we cluster mutants.
To cluster mutants we need a mutation testing tools.
Three mutation testing tools for Java are compared and one is selected for further use in this research.
The first part consists of a white box approach and the second part of a black box approach.
Our research uses the mutation testing tool PIT.
We identified characteristics to represent the mutants such that we can cluster them.
For our white box approach we used hierarchical clustering and for our black box approach we trained the fuzzy c-means  model.
We calculated a weighted mutation score and compared this with the mutation score of a full set of mutants executed.
Results show that with hierarchical clustering we can reduce the amount of mutants executed while maintaining the effectiveness.
The clusters generated by the machine learning model were less accurate and showed a significant decrease in effectiveness.
\end{abstract}

\begin{IEEEkeywords}

\end{IEEEkeywords}

\section{Introduction}
The high cost requirement is often a barrier for adopting mutation testing \cite{Pizzoleto2019}.
A lot of techniques and methods have been developed to improve the performance, however, most of these approaches are not as effective as mutation testing a full set of mutants\cite{Pizzoleto2019,Yao2014}. 
\newline
A technique to reduce the number of mutants or the number of test case executions is mutation clustering.
Mutation clustering aims to reduce the amount of mutants to be executed by clustering mutants\cite{Ma2016,Yu2019PossibilityScope}.
The clustering of mutants has been researched with promising results\cite{Ji2009,Wilinski2015,Ma2016}. For example Ma et al., \cite{Ma2016} clustered the mutants for expressions. 
However in their paper they indicate that the choosing expressions as the domain of their cluster is a limitation.
A recurring problem in the research of mutation clustering is to determine the means of domains as the centre of a cluster\cite{Ji2009,Wilinski2015,Ma2016,Wei2021SpectralTesting}.
\newline
To reduce the cost of mutation testing, we try to find a solution that can cluster mutants while maintaining the accuracy of the same complete set of mutants. 
Our goal is to cluster any mutant that is generated within the Java programming language.
This research aims to remove the scoping limitations present in existing research.
There should be no requirements for which mutants can be clustered.
We do this with two different techniques: qualitative and quantitative.
We devise a white and black box approach for clustering mutants.
Our white box approach contains a qualitative analysis on mutants and a methodology to cluster them.
The black box approach makes use of a machine learning model to cluster mutants.
We structure our research on the following research questions:
\newline
\textbf{Research Question 1}: What set of characteristics can we identify for clustering generated mutants to reduce the amount executed, while maintaining effectiveness?
\newline
\textbf{Research Question 1.1}: How do the existing mutation testing tools for Java compare to each other?
\newline
\textbf{Research Question 2}: How can we train a machine learning model to recognize and cluster generated mutants to reduce the amount executed, while maintaining effectiveness?

\subsection{Contributions}
Our research makes the following contributions:
\begin{enumerate}
 \item A white box methodology to cluster generated mutants based on chosen characteristics to reduce the cost of mutation testing.
  \item A black box methodology to cluster generated mutants based on chosen characteristics to reduce the cost of mutation testing.
 \item A proof of concept which implements the qualitative methodology chosen and elaborated within the thesis.
 \item A proof of concept which implements the quantitative approach to cluster mutants and is elaborated within the thesis. 
\end{enumerate}
\subsection{Outline}

%%%%%% DO RESEARCH PART HERE %%%%%%%%

\section{Selecting a mutation testing tool}
We reviewed literature on three different mutation testing tools[CITE OWN PAPER HERE].
However the most recent study is from 2017.
At the time of writing this paper these studies are at least four years old. 
Some of the tools were updated or are still under active development\cite{pit-releases,Major}.
There are three candidate tools; Major, MuJava and PIT.
We extend the existing mutant operator comparison with the operators that have been added since the publishing date of the existing literature.
Additionally we add an overview of the overlapping mutant operators for the new mutant operators.


\todo{show tables here? maybe a summarized version of this?}
\todo{the results can be found in ref to paper??}

MuJava is not actively maintained and has not been updated in the last few years.
It does not support JUnit 4 and all versions of TestNG\cite{mujava}.
These are test engines and are crucial for executing tests.
MuJava also does not support source projects with java version 1.7 or higher\cite{mujava}.
Conforming to these requirements would reduce the set of projects we could use in or experiments.
\newline
While Major supports JUnit 4,
we did not succeed in generating mutants with this tool[CITE OWN PAPER HERE].
It would be too time consuming to customize all source projects to work with Major.
Furthermore there is not much documentation available for this tool.
\newline
PIT targets the industry, is actively maintained and is open source\cite{Kintis2016AnalysingStudy}.
It supports Maven, Gradle, has a command line interface and has a faster execution time than the other tools.
For example PIT provides a plugin system in which you can inject your own code in various stages of mutation testing process\cite{pit}.
PIT also generated significantly more mutants in every project we have tested[CITE OWN PAPER HERE].
Based on the information presented in this chapter we decide that PIT is the best choice for generating and executing mutants for this research.


\section{Qualitative clustering approach}
\section{Quantitative clustering approach}
\section{validation}


\section{Related work}
\label{sec:rw}

\section{Conclusion}
\label{sec:conclusion}

\printbibliography[heading=bibintoc]

\end{document}
