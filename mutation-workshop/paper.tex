\documentclass[conference]{IEEEtran}
\IEEEoverridecommandlockouts
% The preceding line is only needed to identify funding in the first footnote. If that is unneeded, please comment it out.
% \usepackage{cite}
\usepackage{amsmath,amssymb,amsfonts}
\usepackage{algorithmic}
\usepackage{graphicx}
\usepackage{textcomp}
\usepackage{xcolor}
\usepackage{makecell}
\usepackage{caption}
\usepackage{subcaption}
\usepackage{hyperref}

\usepackage[flushleft]{threeparttable}
\usepackage{tabularx}

\usepackage[disable]{todonotes}
\usepackage{enumitem}

\usepackage[backend=biber, style=ieee, citestyle=numeric-comp,uniquelist=false, maxcitenames=2, maxbibnames=9, mincitenames=1]{biblatex}
\bibliography{../refs/main}


\usepackage{listings}
\lstset{
	basicstyle=\footnotesize,        % the size of the fonts that are used for the code
	breakatwhitespace=false,         % sets if automatic breaks should only happen at whitespace
	breaklines=true,                 % sets automatic line breaking
	captionpos=b,                    % sets the caption-position to bottom
	frame=single,                    % adds a frame around the code
	language=Java,                 % the language of the code
	keywordstyle=\bf,
	tabsize=2                       % sets default tabsize to 2 spaces
}


\def\BibTeX{{\rm B\kern-.05em{\sc i\kern-.025em b}\kern-.08em
    T\kern-.1667em\lower.7ex\hbox{E}\kern-.125emX}}

\def\code#1{\texttt{#1}}
\begin{document}

\title{Mutation testing: Clustering the mutants}


\author{
\IEEEauthorblockN{Ana Oprescu}
\IEEEauthorblockA{\textit{Parallel Computing Systems} \\
\textit{University of Amsterdam}\\
Amsterdam, the Netherlands \\
a.m.oprescu@uva.nl}

\and
\IEEEauthorblockN{Rasjaad Basarat}
\IEEEauthorblockA{\textit{Parallel Computing Systems} \\ %% TODO: CHANGE THIS
\textit{University of Amsterdam}\\
Amsterdam, the Netherlands \\
rasjaad@nivero.io}
\and
\IEEEauthorblockN{Thomas Biesaart}
\IEEEauthorblockA{\textit{Parallel Computing Systems} \\ %% TODO: CHANGE THIS
Amsterdam, the Netherlands \\
thomas@emailhere.com}
}

\maketitle
% \todo[inline]{FROM REVIEWER: Make sure to be consistent with singular/plural (e.g. mutation operators will result in mutants, not just a mutant)}
% \todo[inline]{FROM REVIEWER:  I don't quite follow the discussion of the execution time problem. After all, even in a decentralized approach I can run multiple approaches on the same hardware and thus make a fair comparison. I suppose the point you're trying to make is comparison *across multiple papers/studies"? Please clarify.}
\begin{abstract}
Mutation testing is a computationally expensive technique that requires the ability to run tests.
Different techniques have been proposed to reduce the cost of mutation testing.
One of these techniques is the clustering of mutants.
By clustering mutants we can execute less mutants to reduce the cost.
Different techniques have been proposed to reduce the cost of mutation testing.
One of these techniques is the clustering of mutants.
Our research consists of two parts in which we cluster mutants.
To cluster mutants we need a mutation testing tools.
Three mutation testing tools for Java are compared and one is selected to use for this research
The first part consists of a white box approach and the second part of a black box approach.
Our research uses the mutation testing tool PIT.
We identified characteristics to represent the mutants such that we can cluster them.
For our white box approach we used hierarchical clustering and for our black box approach we trained the fuzzy c-means  model.
We calculated a weighted mutation score and compared this with the mutation score of a full set of mutants executed.
Results show that with hierarchical clustering we can reduce the amount of mutants executed while maintaining the effectiveness.
The clusters generated by the machine learning model were less accurate and showed a significant decrease in effectiveness.
\end{abstract}

\begin{IEEEkeywords}

\end{IEEEkeywords}

\section{Introduction}
\subsection{Contributions}
\subsection{Outline}

%%%%%% DO RESEARCH PART HERE %%%%%%%%

\section{Comparing mutation testing tooling}
\section{Qualitative clustering approach}
\section{Quantitative clustering approach}
\section{Experiment design}



\section{Discussion}

\section{Related work}
\label{sec:rw}

\section{Conclusion}
\label{sec:conclusion}

\printbibliography[heading=bibintoc]

\end{document}
